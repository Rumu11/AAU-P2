\title{Projektbeskrivelse}
\documentclass[12pt]{Article}
\begin{document}

\maketitle
\textit{
Selvom alt fra skovbrug til dataindtastning altid har været en vigtig del af den menneskelige tilværelse, så lægger menneskers interesser og kultur langt ud over nyttig problemløsning. }
\textit{
Specialiseret maskinintelligens er i rivende udvikling inden for alle områder, og derfor er menneskets forhold til maskiner vigtigere end nogensinde.}
\textit{
Selvom en generel AI, hvis intelligens er sammenlignelig med menneskers, stadig ligger langt uden for hvad man kan skabe med den nuværende teknologi, er det, som det altid har været, vigtigt også at arbejde med "unyttige" men for mennesker interessante opgaver.}
\textit{
Musik findes på en eller anden form i alle kendte kulturer, og fungerer som et samlingspunkt for folk. Derudover er musikteori velbeskrevet, og de faste former som musik følger er mulige at efterligne for computere.}
\textit{
Computerkomponeret musik baseret på algoritmer og/eller maskinindlæring er derfor et relevant emne, og kan  forhåbentlig have noget at sige for fremtiden for maskinintelligens, både opbygning og opfattelse.}





Derfor vil vi undersøge, om der er et område af computergnereret musik der kan forbedres, eller om gamle teknikker kan bruges i nye sammenhænge.

Beskrivelse af musik varierer fra kultur til kultur, men det er de samme grundlæggende begreber der går igen. Med en række fundamentale begreber kan man beskrive de fleste typer musik:

\textbf{Pitch:} En subjektiv oplevelse af, at visse lyde er lysere eller dybere end andre af samme slags. Pitches kan beskrives med mange forskellige nodesystemer.

\textbf{Melodi:} En serie af noder der udgør en del af musikken.

\textbf{Harmoni:} Noder der spilles samtidig eller i umiddelbar rækkefølge, og som opleves som sammenhængende.

\textbf{Rytme:} Et regelmæssigt mønster der angiver hvornår musikkens lyde forekommer.

\textbf{Musikkens tekstur:} Et mål for mængden af parallelle tonemønstre og taktslag, og disses kompleksitet.

\textbf{Klangfarve:} Lydens overordnede kvalitet, uafhængig af tone. Det er klangfarven der definerer forskellen mellem violin og klaver.

\textbf{Form:} Musikkens struktur i forhold til nodernes placering. Blandt andet et mål for hvor hurtigt musikken gentager sig og hvor mange variationer der indføres.

Det er disse eller lignende parametre som et computerprogram skal tage stilling til, hvis det skal producere noget der ligner musik, og programmet skal derfor kunne følge de samme regler som komponister bruger når de skriver musikken.



Dette kan programmet gøre på den traditionelle måde, hvilket vil sige, danne musikken fra bunden vha. en række algoritmer og et element af tilfældighed. En anden mulighed er at programmet kan gøre brug af eksisterende musik og ud fra det danne ny musik.

Der findes flere forskellige metoder på hvordan komponering af musik over computeren kan gennemføres, nogle af disse metoders teori er følgende:

Ved at se på tidligere musik kan et computerprogram danne statistik for, hvilke noder der sædvanligvis kommer i rækkefølge. Dette kan for eksempel modelleres med \textbf{Markov-kæder} eller \textbf{Transitionsnoder}. Ved at trække på den indsamlede data kan et program danne lignende musik ved at kopiere små stykker af melodierne, som sættes ind i nye sammenhænge og variationer.

Alternativt kan musikken være baseret på generel musikteori, for eksempel styret med \textbf{generativ grammatik}, hvor i forvejen definerede regler danner rammen for programmets ellers tilfældige komposition.

Derudover findes der visse mønstre som i sig selv kan oversættes til musik og give acceptable resultater. Blandt eksemplerne er \textbf{fraktaler} eller andet fra \textbf{kaos teori }samt \textbf{cellular automata}.

Til sidst kan nævnes \textbf{genetiske algoritmer }som simulerer evolution, hvor det altid er de bedst tilpassede individer der overlever, hvor det for eksempel er den melodi der bedst passer til de kriterier der er sat. De melodier der opfylder kravene dårligst, bliver fjernet fra den virtuelle pulje, mens de overlevende melodier enten  muterer eller arver noget fra en af de andre melodier. hvorefter at simulationen igen udregner hvor tilpasset de nye melodier opfylder kravet for overlevelse. 



Målet skulle være at forbedre computergenereret musik på et eller flere punkter og på den måde komme tættere på noget der kan siges at være ægte maskingenereret kunst, eller i det mindste tilføje endnu en dimension til diskussionen om hvad der kan og ikke kan være kunst. Fra et menneskeligt perspektiv er spørgmålet, hvordan computergnereret musik kunne formidles, hvordan det ville blive modtaget, og om det overhovedet kunne få en kulturel betydning.

\end{document}